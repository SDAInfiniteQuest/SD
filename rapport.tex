\documentclass{article}
\usepackage[francais]{babel}
\usepackage[utf8]{inputenc}
\begin{document}

\title{Rapport de systeme distribue}
\author{Pallamidessi Joseph \\
        Simon Andreux
        }
\maketitle

\section{Intro} % (fold)
\label{sec:Intro}
  \paragraph{} % (fold)
  \label{par:}
    Le projet a ete realise en C89 standard avec les bibliotheques suivantes: rpc pour la partie
    reseaux/distribue,SDL pour la partie graphique et openMP pour le parallelisme. Les
    programme client et serveur compilent sans warning et n'ont aucune fuites memoires.
  % paragraph  (end)
% section Intro (end)

\section{Mise en place} % (fold)

\label{sec:Mise en place}
  \paragraph{} % (fold)
  \label{par:}
    Le programme se compile en avec un simple appel a make dans le repertoire racine du
    projet.Une commande make clean est egalement possible.\\
    Les serveur se lance en precisant 2 argument en entree leurs procnum et leurs versnum, 
    mais doivent etre scritement les meme,on precise aussi le nombre de threads voulu.\\
    Le client se lance en precisant l'adresse IP (unique) des serveur et leurs nombre. \\
    Pour lancer un run,on peut utiliser le script run fourni: "run \[nombre de serveurs voulu\]" ou
    bien manuellement.\\
    ex. de lancement manuelle: serveur 1 1 4& \\
                               serveur 2 2 4& \\
                               client localhost 2\\
  % paragraph  (end)
% section Mise en place (end)

\section{Structure de donnee et initialisation} % (fold)
\label{sec:Structure de donnee}
  \paragraph{} % (fold)
  \label{par:}
    Les structure utilisees sont les suivantes : des population ayant des allocation dynamique
    d'adn, ayant elles-meme des allocation dynamique de displacement. Le fait d'avoir des
    allocation dynamique de displacement avait un sens pour gerer le croissement d'adn de
    taille differentes.L'algorithme presente ici utilise des taille fixe.\\
    Les displacement se font dans 8 direction et on des taille variable (de 1 a 30 pixel),
    tous comme le parametre (#define) NB\_DISPLACEMENT renseigne le nomdre de displacement
    utilises par adn.
  % paragraph  (end)
    \paragraph{paragraph name} % (fold)
    \label{par:paragraph name}
      Le "monde" est une matrice represente de maniere lineaire en memoire pour facilite
      sont transfert.
    % paragraph paragraph name (end)
    \paragraph{} % (fold)
    \label{par:}
      L'initialisation des adn est completement aleatoire. Pour se faire une creer des
      displacement de direction aleatoire et de longueur borne par 30 pixel. 
    % paragraph  (end)
% section Structure de donnee (end)

\section{Algorithme genetique } % (fold)
  \paragraph{} % (fold)
  \label{par:}
    Comme dit precedemment l'algorithme genetique gere des population de taille fixe,avec
    des adn de taille fixe. Nous allons maintenant expose etapes par etapes sont
    fonctionnement. Des screenshot de fonctionnement sont disponible en annexe.
  % paragraph  (end)

  \paragraph{} % (fold)
  \label{par:}
    Le point de depart des adn est le milieu du bord gauche et le point d'arrivee est le
    milieu du bord droit. Le parametre (#define) GRID\_SIZE definit la taille pixel du
    monde (il s'agit d'un carre),ici place a 840.
  % paragraph  (end)

    \subsubsection{Evaluation} % (fold)
    \label{ssub:Evaluation}
      \paragraph{} % (fold)
      \label{par:}
      La fonction d'evaluation est la piece centrale du bon fonctionnement de l'algorithme.
      Ici,elle selon certain parametres crucial: nombre de collision, sortie du "monde",
      distance restante entre la fin de chaque adn et l'arrivee.\\
      Nous ne prenons pas en compte la longueur de l'adn pour l'instant pour evite de
      surcharge l'algorithme d'information, met le systeme est en place pour gerer la
      longueurs.
      % paragraph  (end)
      \paragraph{} % (fold)
      \label{par:}
      Les valeurs numerique pour l'evaluation sont : pas de collision +100, -50 par
      collision,100-distance entre l'arrive est le dernier point. Donc l'evaluation
      maxiaml est de 200 dans le cas d'un individu qui ne rencontre pas d'obstacles et qui
      arrive a la fin.
      % paragraph  (end)
    % subsubsection Evaluation (end)

    \subsubsection{Selection} % (fold)
    \label{ssub:Selection}
      \paragraph{} % (fold)
      \label{par:}
        il s'agit d'une fonction relativement simple on selectionne entre 2 population(les
        meilleurs de la generation precedente,et leurs fils que l'on vient de creer) les
        meilleurs adn (meilleurs note) a l'aide d'un quicksort.
      % paragraph  (end)
    % subsubsection Selection (end)
    
    \subsubsection{Crossing-over} % (fold)
    \label{ssub:Crossing-over}
     \paragraph{} % (fold)
     \label{par:}
        On utilise une fonction de crossing-over (reproduction) qui se rapproche de la
        nature. On prend du materiel genetique (ici des displacement) des deux parents
        pour creer le fils. Dans la pratique, mieux vaut prendre plus de materiel au
        meilleurs des deux parent, ce aui est implemente (on prend \%60 du meilleurs).
     % paragraph  (end)
    % subsubsection Crossing-over (end)

    \subsubsection{Mutation} % (fold)
    \label{ssub:Mutation}
      \paragraph{} % (fold)
      \label{par:}
        Chaque adn nouvellement creer peut iavoir des displcement qui mute en un autre a
        hauteurs de 1\%.
      % paragraph  (end)
    % subsubsection Mutation (end)
% section Algorithme genetique (end)

\section{Partie distribue} % (fold)
\label{sec:Partie distribue}
  \paragraph{} % (fold)
  \label{par:}
    La distribution est deploye comme suit: le client centrale envoie la matrix du monde a
    tout les serveurs, puis demande a tout les serveur de lancer leurs calcul de genetique.
    Chaque serveur envoyent ses meilleurs individus calcules en local. Le client
    selectionne alors a travers tout les meilleurs local de chaque serveur les meilleurs
    global et les renvoye a tous les serveurs pour qu'ils les incorpore a toutes les
    sous-population qu'ils gereent.

  % paragraph  (end)
% section Partie distribue (end)

\section{Parallelisme} % (fold)
\label{sec:Parallelisme}
  \paragraph{} % (fold)
  \label{par:}
    Chaque serveur lance dans des threads l'algorithme de recherche sur n
    sous\_population, et a la meme maniere que pour la distribution les meilleurs individu
    de chaque sous-population sont selectionnes et reinjectes dans toutes les
    sous-populations.
  
  % paragraph  (end)
% section Parallelisme (end)

\section{Partie graphique} % (fold)
\label{sec:Partie graphique}

  \subsubsection{Choix de la librairie} % (fold)
  \label{ssub:Choix de la librairie}
    \paragraph{} % (fold)
    \label{par:}
      Nous avons choisi de réaliser d'implémenter l'affichage à l'aide la
      librairie SDL à cause de sa simplicité qui nous a semblé suffisante pour
      le projet.
      Toutefois, nous avons rencontré de légères difficultés quant à
      l'implémentation, la librairie ne disposant pas d'une fonction putpixel,
      il a fallu la recoder.
    % paragraph  (end)

  % subsubsection Choix de la librairie (end)

  \subsubsection{Adaptation a l'algorithme genetique} % (fold)
  \label{ssub:Adaptation a l'algorithme genetique}
    \paragraph{} % (fold)
    \label{par:}
      Une fonction displayi\_world crée une représentation du monde et des cercles
      en les mappant directement sur les pixels.
      Une autre fonction display\_adn affiche le meilleur adn parmi tous les
      adns en bleu.
    % paragraph  (end)

  % subsubsection Adaptation a l'algorithme genetique (end)


  \subsubsection{Amelioration possible} % (fold)
  \label{ssub:Amelioration possible}
    \paragraph{} % (fold)
    \label{par:}
      Une idée intéressante aurait été d'afficher les premiers chemins retenues
      avec une couleur différente et de teinter les aretes des chemins selon
      leur proximité avec le point d'arrivée.
    % paragraph  (end)
  % subsubsection Amelioration possible (end)
% section Partie graphique (end)


\section{Remarque et amelioration possible} % (fold)
\label{sec:Remarque et amelioration possible}

% section Remarque et amelioration possible (end)

        Différentes remarques par rapport à l'algorithme génétique.
        
        \paragraph{} % (fold)
        \label{par:}
        
        -1./ La fonction d'évaluation dépend de différents paramètres pour
        fonctionner les malus et les bonus relatifs aux adns sont dépendants des
        parmètres utilisés pour créer le monde. Par conséquent, il pourrait être
        intéressant de calculer les bonus et les malus depuis les paramètres
        utilisés pour créer le monde.
        % paragraph  (end)
        
        \paragraph{} % (fold)
        \label{par:}
        
        -2./ Il existe un problème inhérent à l'utilisation de paramètres pour la
        fonction d'évaluation de l'algorithme génétique. L'efficacité est liée à
        l'utilisation des paramètres, des paramètres mal choisis rendent la
        fonction parfaitement inefficace. Ainsi, la fonction avec un trop fort
        malus tend à créer des "pelotes de laines" et rester bloquée dans une
        zone sans collision. En réduisant trop le malus aux collisions, on
        obtient des collisions des collisions dans la version finale.
        Une amélioration éventuelle consisterait à générer dynamiquement les
        paramètres de malus:\\
                -en tenant compte du monde (diametre des cercles, nombre de
                cercles,densité...)\\
                -en utilisant l'algorithme génétique pour choisir une valeur minimale des bonus/malus tel que l'on puisse générer en un
                cycle des adns sans
                collisions.


        % paragraph  (end)


\section{Repartition des tache} % (fold)
\label{sec:Repartiotion des tache}
Joseph s'est occuper de la partie genetique,distribue et parallelisme.\\
Simon de la partie affichage graphique,creation du monde et divers bugs.
% section Repartiotion des tache (end)


\end{document}
\end{article}
